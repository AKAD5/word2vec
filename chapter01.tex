\chapter{拉格朗日对偶性}
\section{约束优化问题}
设目标函数$f(x)$是定义在$R^n$上的连续可微函数,在变量$x$满足某些约束的条件下求f(x)的最优值的问题称为约束优化问题。如:
\begin{align*}
    & \underset{x}{\min}\ f(x) \\
    s.t.\quad & g_i(x) \le 0, i=1, 2, ..., m; \\
         & h_j(x) = 0, j=1, 2, ..., l;
\end{align*}
表示在满足等式和不等式约束条件下,求得函数f(x)的最小值。这里,函数g(x)和h(x)均是定义在$R^n$空间上的连续可微函数。
我们引入系数参数$\alpha$,$\beta$,定义如下的Lagrange函数:
$$
L(x,\alpha,\beta) = f(x) + \sum_{i=1}^m \alpha_i g_i(x) + \sum_{j=1}^l \beta_j h_j(x)
$$
定义如下函数:
$$
    \theta_P(x) = \underset{\alpha,\beta;\alpha_i \ge 0}{\max} L(x,\alpha,\beta)
$$
$\theta_P(x)$具有如下性质:
$$\theta_P(x)=
\begin{cases}
    f(x)\quad &\mbox{如果$x$满足约束条件};\\
    \infty &\mbox{如果$x$不满足约束条件};
\end{cases}
$$
那么原始优化问题的等价表示是:
$$ \underset{x}{min}\ \theta_P(x) $$
这样
$$\underset{x}{min} \underset{\alpha,\beta;\alpha_i \ge 0}{\max} L(x,\alpha,\beta)$$表示Lagrange函数的极小极大问题,
$$ p^*=\underset{x}{min}\ \theta_P(x) $$
是原始问题的最优值。

\section{约束优化问题的对偶问题}
定义如下函数:
$$
    \theta_D(\alpha,\beta) = \underset{x}{\min}\ L(x,\alpha,\beta)
$$
于是:
$$ \underset{\alpha,\beta;\alpha_i \ge 0}{\max}\ \theta_D(\alpha,\beta) $$
称为原始问题的对偶问题。\\
这样
$$\underset{\alpha,\beta;\alpha_i \ge 0}{\max} \underset{x}{\min}\ L(x,\alpha,\beta)$$
表示Lagrange函数的极大极小问题,
$$ d^*=\underset{\alpha,\beta;\alpha_i \ge 0}{\max} \theta_D(\alpha,\beta) $$
是对偶问题的最优值。

\section{一些定理}
\begin{theorem}{}{}
    若原始问题和对偶问题都有最优值,$p^*$是原始问题的最优值,$d^*$是对偶问题的最优值,则:
    $$ d^* \le p^* $$
\end{theorem}
\begin{theorem}{}{}
    若$x^*$是原始问题的可行解,$\alpha^*, \beta^*$是对偶问题的可行解,且满足
    $$p^*=d^*$$
    那么$x^*$,$\alpha^*, \beta^*$是原始问题和对偶问题的最优解。
\end{theorem}
\begin{theorem}{}{}
    若函数$f(x)$和$g_i(x)$是凸函数,$h_i(x)$是仿射函数,并且不等式约束$g_i(x)\le0$严格成立,则存在$x^*$,$\alpha^*, \beta^*$,使得$x^*$是原始问题的最优解,$\alpha^*, \beta^*$是对偶问题的最优解,且:
        $$ p^* = d^* = L(x^*,\alpha^*,\beta^*) $$
\end{theorem}
\begin{theorem}{}{}
    若函数$f(x)$和$g_i(x)$是凸函数,$h_i(x)$是仿射函数,并且不等式约束$g_i(x)\le0$严格成立,$x^*$,$\alpha^*,\beta^*$分别是原始问题和对偶问题的最优解的充要条件是:
    $x^*$,$\alpha^*,\beta^*$需满足如下的KKT(Karush-Kuhn-Tucker)条件(定常方程式,原始可行性,对偶可行性,互补松弛性)。
    \begin{gather*}
        \nabla_x L(x^*,\alpha^*,\beta^*) = 0 \\
        g_i(x^*) \le 0,\ i=1, 2, ..., m \\
        h_j(x^*) = 0,\ j=1, 2, ..., l \\
        \alpha_i^* \ge 0,\ i=1, 2, ..., m \\
        \alpha_i^*g_i(x^*) = 0,\ i=1, 2, ..., m
    \end{gather*}
\end{theorem}
